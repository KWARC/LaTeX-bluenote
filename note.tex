\documentclass{bluenote}
\usepackage[show]{ed}
\usepackage{listings}
\lstset{basicstyle=\sf,columns=fullflexible}
\usepackage[hyperref,backend=bibtex,style=alphabetic]{biblatex}
\addbibresource{../bibs/kwarcpubs.bib}
\addbibresource{../bibs/extpubs.bib}
\addbibresource{../bibs/kwarccrossrefs.bib}
\addbibresource{../bibs/extcrossrefs.bib}
\usepackage{hyperref}

\title{Blue Notes -- Writing down $\epsilon$-baked ideas in {\LaTeX}}
\author{Michael Kohlhase\\
  Computer Science, FAU Erlangen-N\"urnberg\\
  \url{http://kwarc.info/kohlhase}}

\begin{document}
\maketitle
\begin{abstract}
  This is a skeleton for a blue note.
\end{abstract}

\section{Introduction}\label{sec:intro}

Blue notes are a great way to preserve your ideas. 

\section{Workflow}\label{sec:main}

You choose a repository for your blue notes, e.g. \url{http://gl.kwarc.info/jdoe/blue},
and make \lstinline|git subrepo|s for the KWARC bibs~\cite{KBibs:on} and the
\lstinline|bluenote| class~\cite{BlueNote:on} (see the \lstinline|README.md| files for
details), then you can copy this paper into a subdir of \lstinline|blue|, and just start
writing.

It is a good idea to write from the middle (i.e. intro/conclusion last) and use the
\lstinline[language=TeX]|\ednote| macro~\cite{ed:on} to mark up editorial comments to
yourself -- e.g. what you still need to introduce for the text to make sense.

\section{Conclusion}\label{sec:concl}

Blue notes are a great way to preserve your ideas. The \lstinline|bluenote| class makes it
easier for {\LaTeX} users. 

\begin{sloppypar}
\printbibliography
\end{sloppypar}
\end{document}

%%% Local Variables: 
%%% mode: latex
%%% TeX-master: t
%%% End: 
